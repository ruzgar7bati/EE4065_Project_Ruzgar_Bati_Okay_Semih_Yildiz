\section*{Abstract}

This report presents the implementation of image processing and basic computer vision methods on the ESP32-CAM, a low-cost microcontroller with an integrated camera and limited computational resources. Due to these constraints, all algorithms are designed to be efficient and suitable for embedded systems.

The project includes three main components. First, a histogram-based thresholding method is implemented to detect bright objects on dark backgrounds. Second, a handwritten digit detection system is developed using a YOLO-based model trained to recognize the digits 0, 4, and 7. Third, image resizing operations are implemented to support both upsampling and downsampling with non-integer scale factors such as 1.5× and 2/3.

All implementations rely on integer arithmetic instead of floating-point operations to ensure compatibility with the ESP32-CAM hardware. The development follows a two-stage workflow: algorithms are first designed and tested in Python on a PC, then ported and optimized in C/C++ for execution on the ESP32-CAM. Experimental results show that efficient integer-based methods can achieve practical image processing performance on resource-constrained hardware. The thresholding algorithm successfully extracts target objects, the YOLO model detects handwritten digits with reasonable accuracy, and the resizing operations produce acceptable visual results for embedded applications.
