\section{Question 1: Complete Code Listings}
\label{appendix:q1}

\subsection{Python Implementation}
\label{appendix:q1:python}

\begin{lstlisting}[style=pythonstyle, caption=Python Implementation for Thresholding (\texttt{python/q1.py})]
from PIL import Image
import matplotlib.pyplot as plt
from pathlib import Path


def rgb_to_grayscale(image):
    """
    Convert RGB image to grayscale using integer arithmetic.
    Formula matches embedded-friendly luminance approximation.
    gray = (30*R + 59*G + 11*B) / 100
    """

    width, height = image.size
    gray = [[0 for _ in range(width)] for _ in range(height)]

    pixels = image.load()

    for y in range(height):
        for x in range(width):
            r, g, b = pixels[x, y]
            gray[y][x] = (30 * r + 59 * g + 11 * b) // 100

    return gray


def extract_bright_pixels_histogram(gray, max_pixels=1000):
    """
    ESP32-friendly adaptive thresholding using histogram accumulation.
    Ensures output contains at most 'max_pixels' bright pixels.
    All other pixels are set to 0.
    """

    height = len(gray)
    width = len(gray[0])

    # 1. Build grayscale histogram
    hist = [0] * 256
    for y in range(height):
        for x in range(width):
            hist[gray[y][x]] += 1

    # 2. Find threshold from brightest to darkest
    cumulative = 0
    threshold = 255
    for intensity in range(255, -1, -1):
        cumulative += hist[intensity]
        if cumulative >= max_pixels:
            threshold = intensity
            break

    # 3. Create binary output image
    output = [[0 for _ in range(width)] for _ in range(height)]

    selected = 0
    for y in range(height):
        for x in range(width):
            if gray[y][x] >= threshold and selected < max_pixels:
                output[y][x] = 255
                selected += 1
            else:
                output[y][x] = 0

    return output, threshold


def visualize(gray, binary):
    """
    Visualization for PC validation only.
    Not part of embedded logic.
    """

    plt.figure(figsize=(10, 4))

    plt.subplot(1, 2, 1)
    plt.title("Grayscale Image")
    plt.imshow(gray, cmap="gray")
    plt.axis("off")

    plt.subplot(1, 2, 2)
    plt.title("Binary Output (≤ 1000 pixels)")
    plt.imshow(binary, cmap="gray")
    plt.axis("off")

    plt.tight_layout()
    plt.show()


def main():
    # Load image (RGB) relative to this script's folder
    base_dir = Path(__file__).resolve().parent
    img_path = base_dir / "question1_images" / "reference_taken_from_phone.jpg"
    image = Image.open(img_path).convert("RGB")

    # For Python testing only: resize to match ESP32 resolution
    target_size = (96, 96)
    image = image.resize(target_size, Image.BILINEAR)

    # Convert to grayscale (ESP32-style)
    gray = rgb_to_grayscale(image)

    # Apply histogram-based thresholding
    binary, threshold = extract_bright_pixels_histogram(
        gray, max_pixels=1000
    )

    print("Selected threshold intensity:", threshold)

    # Visualization (PC only)
    visualize(gray, binary)


if __name__ == "__main__":
    main()
\end{lstlisting}

\subsection{ESP32-CAM Implementation}
\label{appendix:q1:esp32}

\begin{lstlisting}[style=cstyle, caption=ESP32 CAM Thresholding Code (\texttt{esp32\_cam\_link/esp32\_cam\_q1/esp32\_cam\_q1.ino})]
#include "esp_camera.h"
#include <string.h>

// ===== AI Thinker ESP32-CAM pin map =====
#define PWDN_GPIO_NUM     32
#define RESET_GPIO_NUM    -1
#define XCLK_GPIO_NUM      0
#define SIOD_GPIO_NUM     26
#define SIOC_GPIO_NUM     27
#define Y9_GPIO_NUM       35
#define Y8_GPIO_NUM       34
#define Y7_GPIO_NUM       39
#define Y6_GPIO_NUM       36
#define Y5_GPIO_NUM       21
#define Y4_GPIO_NUM       19
#define Y3_GPIO_NUM       18
#define Y2_GPIO_NUM        5
#define VSYNC_GPIO_NUM    25
#define HREF_GPIO_NUM     23
#define PCLK_GPIO_NUM     22

#define FLASH_GPIO        4

#define WIDTH  96
#define HEIGHT 96
#define MAX_PIXELS 1000

void setup() {
  Serial.begin(921600);
  delay(2000);

  pinMode(FLASH_GPIO, OUTPUT);
  digitalWrite(FLASH_GPIO, LOW);

  camera_config_t config;
  config.ledc_channel = LEDC_CHANNEL_0;
  config.ledc_timer   = LEDC_TIMER_0;
  config.pin_d0 = Y2_GPIO_NUM;
  config.pin_d1 = Y3_GPIO_NUM;
  config.pin_d2 = Y4_GPIO_NUM;
  config.pin_d3 = Y5_GPIO_NUM;
  config.pin_d4 = Y6_GPIO_NUM;
  config.pin_d5 = Y7_GPIO_NUM;
  config.pin_d6 = Y8_GPIO_NUM;
  config.pin_d7 = Y9_GPIO_NUM;
  config.pin_xclk = XCLK_GPIO_NUM;
  config.pin_pclk = PCLK_GPIO_NUM;
  config.pin_vsync = VSYNC_GPIO_NUM;
  config.pin_href = HREF_GPIO_NUM;
  config.pin_sscb_sda = SIOD_GPIO_NUM;
  config.pin_sscb_scl = SIOC_GPIO_NUM;
  config.pin_pwdn = PWDN_GPIO_NUM;
  config.pin_reset = RESET_GPIO_NUM;

  config.xclk_freq_hz = 20000000;
  config.pixel_format = PIXFORMAT_GRAYSCALE;
  config.frame_size   = FRAMESIZE_96X96;
  config.fb_count     = 1;

  esp_camera_init(&config);
}

void loop() {
  digitalWrite(FLASH_GPIO, HIGH);   // Flash ON
  delay(20);                        // allow light to stabilize

  camera_fb_t *fb = esp_camera_fb_get();
  digitalWrite(FLASH_GPIO, LOW);    // Flash OFF

  if (!fb) return;

  // Histogram
  uint32_t hist[256] = {0};
  for (int i = 0; i < fb->len; i++) {
    hist[fb->buf[i]]++;
  }

  // Threshold selection
  uint32_t sum = 0;
  uint8_t threshold = 255;
  for (int i = 255; i >= 0; i--) {
    sum += hist[i];
    if (sum >= MAX_PIXELS) {
      threshold = i;
      break;
    }
  }

  // Binary output
  static uint8_t binary[WIDTH * HEIGHT];
  uint32_t selected = 0;

  for (int i = 0; i < fb->len; i++) {
    if (fb->buf[i] >= threshold && selected < MAX_PIXELS) {
      binary[i] = 255;
      selected++;
    } else {
      binary[i] = 0;
    }
  }

  // Send to PC
  uint8_t sync[2] = {0xAA, 0x55};
  uint32_t size = fb->len;

  Serial.write(sync, 2);
  Serial.write((uint8_t*)&size, 4);
  Serial.write(binary, size);

  esp_camera_fb_return(fb);

  delay(1000);   // 1 frame per second (stable & safe)
}
\end{lstlisting}

\subsection{Python Connector Script}
\label{appendix:q1:receiver}

\begin{lstlisting}[style=pythonstyle, caption=Python Receiver Script (\texttt{esp32\_cam\_link/esp32\_cam\_q1/receive.py})]
import serial
import struct
import numpy as np
from PIL import Image

PORT = "COM4"
BAUD = 921600

WIDTH = 96
HEIGHT = 96

ser = serial.Serial(
    PORT,
    BAUD,
    timeout=5,
    dsrdtr=False,
    rtscts=False
)

ser.setDTR(False)
ser.setRTS(False)

img_count = 0
print("Receiving binary images... Press Ctrl+C to stop.")

try:
    while True:
        # Sync
        if ser.read(1) != b'\xAA':
            continue
        if ser.read(1) != b'\x55':
            continue

        # Length
        size_bytes = ser.read(4)
        size = struct.unpack("<I", size_bytes)[0]

        # Data
        data = ser.read(size)
        if len(data) != size:
            print("Incomplete frame")
            continue

        img = np.frombuffer(data, dtype=np.uint8)
        img = img.reshape((HEIGHT, WIDTH))

        image = Image.fromarray(img, mode='L')
        filename = f"binary.png"
        image.save(filename)

        print(f"Saved {filename}")
        img_count += 1

except KeyboardInterrupt:
    print("\nStopped by user.")
    ser.close()
\end{lstlisting}
